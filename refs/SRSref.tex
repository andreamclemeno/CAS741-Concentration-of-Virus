@misc{who, 
title="HIV/AIDS", 
url="https://www.who.int/news-room/fact-sheets/detail/hiv-aids", journal={World 
Health Organization}, publisher={World Health Organization}} 
 }
@misc{cdc, title={Types of Providers}, 
url={https://www.hiv.gov/hiv-basics/starting-hiv-care/find-a-provider/types-of-providers}, 
journal={HIV.gov}, author={Content Source: CDC's HIV Treatment WorksDate last 
updated: May 21, 2018}, year={2018}, month={Nov}}

@incollection{BURRELL201739,
title = "Chapter 4 - Virus Replication",
editor = "Christopher J. Burrell and Colin R. Howard and Frederick A. Murphy",
booktitle = "Fenner and White's Medical Virology (Fifth Edition)",
publisher = "Academic Press",
edition = "Fifth Edition",
address = "London",
pages = "39 - 55",
year = "2017",
isbn = "978-0-12-375156-0",
doi = "https://doi.org/10.1016/B978-0-12-375156-0.00004-7",
url = "http://www.sciencedirect.com/science/article/pii/B9780123751560000047",
author = "Christopher J. Burrell and Colin R. Howard and Frederick A. Murphy",
keywords = "Virus growth, replication cycle, penetration, uncoating, satellite 
viruses, viroids, genetic diversity, virus assays",
abstract = "Understanding the molecular events accompanying virus replication is 
essential for the proper understanding and control of all virus diseases. The 
virus replication cycle generates new viral genomes and proteins in sufficient 
quantities to ensure propagation of the viral genome; this requires that the 
extracellular viral genome is protected from enzymatic degradation and can be 
introduced into further target cells for further rounds of replication. The 
initial recognition between virus and host is more complex than originally 
supposed and may involve more than one cellular receptor. A critical first 
intracellular step is the generation of viral mRNA by one of a limited number of 
strategies first described by David Baltimore. Lacking ribosomes, viruses have 
no means of producing protein and are reliant on the host cell for protein 
synthesis. Viral proteins are often modified by host cell glycosylation during 
or after virus assembly. Temporal regulation of intracellular events is critical 
in all but the very simplest of viruses, and some form of suppression of the 
host innate immune response is common to nearly all human viruses. Infected 
cells often produce non-infectious particles with incomplete genomes, and these 
defective interfering particles may play a role in pathogenesis. Understanding 
these processes will open up a range of targets for the development of novel 
therapies."
}}
}


@misc{william_2018, title={Definition of T cell}, 
url={https://www.medicinenet.com/script/main/art.asp?articlekey=11300}, 
journal={MedicineNet}, publisher={MedicineNet}, author={William C. Shiel Jr., 
MD}, year={2018}, month={Dec}} 
}

 @misc{hiv.gov, title={What Are HIV and AIDS?}, 
url={https://www.hiv.gov/hiv-basics/overview/about-hiv-and-aids/what-are-hiv-and-aids}, 
journal={HIV.gov}, author={Content Source: HIV.govDate last updated: June 05, 
2020}, year={2020}, month={Jun}} 
}

@misc{wikifig,
   author = "Anonymous Contributors",
   title = "group_5_presentation_3_-_immune_system --- Wiki",
   year = "2019",
   url = 
"https://wiki.mcmaster.ca/LIFESCI_4M03/doku.php?id=group_5_presentation_3_-_immune_system&rev=1554498345",
   note = "[Online; accessed 11-October-2020]"
 }
@article{10.1371/journal.ppat.1002792,
   author = {Roesch, Ferdinand AND Meziane, Oussama AND Kula, Anna AND Nisole, 
Sébastien AND Porrot, Françoise AND Anderson, Ian AND Mammano, Fabrizio AND 
Fassati, Ariberto AND Marcello, Alessandro AND Benkirane, Monsef AND Schwartz, 
Olivier},
   journal = {PLOS Pathogens},
   publisher = {Public Library of Science},
   title = {Hyperthermia Stimulates HIV-1 Replication},
   year = {2012},
   month = {07},
   volume = {8},
   url = {https://doi.org/10.1371/journal.ppat.1002792},
   pages = {1-14},
   abstract = {Author Summary Fever is a complex reaction triggered in response 
to pathogen infection. It induces diverse effects on the human body and 
especially on the immune system. The functions of immune cells are positively 
affected by fever, helping them to fight infection. Fever consists in a 
physiological elevation of temperature and in inflammation. While the role of 
inflammatory molecules on HIV-1 replication has been widely studied, little is 
known about the direct effect of temperature on viral replication. Here, we 
report that hyperthermia (39.5°C) boosts HIV-1 replication in CD4+ T cells. In 
single-cycle infection experiments, hyperthermia increased HIV-1 infection up to 
7-fold. This effect was mediated in part by an increased activation of the HIV-1 
promoter by the viral protein Tat. Our results also indicate that hyperthermia 
may help HIV-1 to reactivate from latency. We also show that the Heat Shock 
Protein Hsp90, which levels are increased at 39.5°C, mediates in a large part 
the positive effect of hyperthermia on HIV-1 infection. Our work suggests that 
in HIV-1-infected patients, fever episodes may facilitate viral replication.},
   number = {7},
   doi = {10.1371/journal.ppat.1002792}
   }
   
@article {Perelson1582,
	author = {Perelson, Alan S. and Neumann, Avidan U. and Markowitz, Martin and 
Leonard, John M. and Ho, David D.},
	title = {HIV-1 Dynamics in Vivo: Virion Clearance Rate, Infected Cell 
Life-Span, and Viral Generation Time},
	volume = {271},
	number = {5255},
	pages = {1582--1586},
	year = {1996},
	doi = {10.1126/science.271.5255.1582},
	publisher = {American Association for the Advancement of Science},
	abstract = {A new mathematical model was used to analyze a detailed set of 
human immunodeficiency virus-type 1 (HIV-1) viral load data collected from five 
infected individuals after the administration of a potent inhibitor of HIV-1 
protease. Productively infected cells were estimated to have, on average, a 
life-span of 2.2 days (half-life t1/2 = 1.6 days), and plasma virions were 
estimated to have a mean life-span of 0.3 days (t1/2 = 0.24 days). The estimated 
average total HIV-1 production was 10.3 {\texttimes} 109 virions per day, which 
is substantially greater than previous minimum estimates. The results also 
suggest that the minimum duration of the HIV-1 life cycle in vivo is 1.2 days on 
average, and that the average HIV-1 generation time{\textemdash}defined as the 
time from release of a virion until it infects another cell and causes the 
release of a new generation of viral particles{\textemdash}is 2.6 days. These 
findings on viral dynamics provide not only a kinetic picture of HIV-1 
pathogenesis, but also theoretical principles to guide the development of 
treatment strategies.},
	issn = {0036-8075},
	URL = {https://science.sciencemag.org/content/271/5255/1582},
	eprint = {https://science.sciencemag.org/content/271/5255/1582.full.pdf},
	journal = {Science}
}

@misc{weisstein, title={Exponential Decay}, 
url={https://mathworld.wolfram.com/ExponentialDecay.html}, journal={from Wolfram 
MathWorld}, author={Weisstein, Eric W}} 
}
%two GG AND GD
@article {Fischer6706,
	author = {Fischer, Ulrike R. and Weisz, Willy and Wieltschnig, Claudia and 
Kirschner, Alexander K. T. and Velimirov, Branko},
	title = {Benthic and Pelagic Viral Decay Experiments: a Model-Based Analysis 
and Its Applicability},
	volume = {70},
	number = {11},
	pages = {6706--6713},
	year = {2004},
	doi = {10.1128/AEM.70.11.6706-6713.2004},
	publisher = {American Society for Microbiology Journals},
	abstract = {The viral decay in sediments, that is, the decrease in benthic 
viral concentration over time, was recorded after inhibiting the production of 
new viruses. Assuming that the viral abundance in an aquatic system remains 
constant and that viruses from lysed bacterial cells replace viruses lost by 
decay, the decay of viral particles can be used as a measure of viral 
production. Decay experiments showed that this approach is a useful tool to 
assess benthic viral production. However, the time course pattern of the decay 
experiments makes their interpretation difficult, regardless of whether viral 
decay is determined in the water column or in sediments. Different curve-fitting 
approaches (logarithmic function, power function, and linear regression) to 
describe the time course of decay experiments found in the literature are used 
in the present study and compared to a proposed {\textquotedblleft}exponential 
decay{\textquotedblright} model based on the assumption that at any moment the 
decay is proportional to the amount of viruses present. Thus, an equation of the 
form dVA/dt = -k {\texttimes} VA leading to the time-integrated form VAt = VA0 
{\texttimes} e-k{\texttimes}t was used, where k represents the viral decay rate 
(h-1), VAt is the viral abundance (viral particles ml-1) at time t (h), and VA0 
is the initial viral abundance (viral particles ml-1). This approach represents 
the best solution for an accurate curve fitting based on a mathematical model 
for a realistic description of viral decay occurring in aquatic systems. Decay 
rates ranged from 0.0282 to 0.0696 h-1 (mean, 0.0464 h-1). Additionally, a 
mathematical model is presented that enables the quantification of the viral 
control of bacterial production. The viral impact on bacteria based on decay 
rates calculated from the different mathematical approaches varied widely within 
one and the same decay experiment. A comparison of the viral control of 
bacterial production in different aquatic environments is, therefore, improper 
when different mathematical formulas are used to interpret viral decay 
experiments.},
	issn = {0099-2240},
	URL = {https://aem.asm.org/content/70/11/6706},
	eprint = {https://aem.asm.org/content/70/11/6706.full.pdf},
	journal = {Applied and Environmental Microbiology}
}
 
  @misc{libretexts_2020, title={14.4: The Change of Concentration with Time 
(Integrated Rate Laws)}, 
url={https://chem.libretexts.org/Bookshelves/General_Chemistry/Map:_Chemistry_-_The_Central_Science_(Brown_et_al.)/14:_Chemical_Kinetics/14.4:_The_Change_of_Concentration_with_Time_(Integrated_Rate_Laws)}, 
journal={Chemistry LibreTexts}, publisher={Libretexts}, author={Libretexts}, 
year={2020}, month={Aug}} 
}
}


 @misc{hobbie_roth_1970, title={Exponential Growth and Decay}, 
url={https://link.springer.com/chapter/10.1007/978-0-387-49885-0_2}, 
journal={SpringerLink}, publisher={Springer, New York, NY}, author={Hobbie, 
Russell K. and Roth, Bradley J.}, year={1970}, month={Jan}} 

