\documentclass{article}
\usepackage{tabularx}
\usepackage{booktabs}


\title{CAS 741: Rate of concentration of Human immunodeficiency virus (HIV-1)}
\author{ Andrea Clemeno   400018541 }
\date{September 21 2020}


\begin{document}

\maketitle

\newpage
\section{Revision History}

\begin{table}[hp]
\caption{Revision History} \label{TblRevisionHistory}
\begin{tabularx}{\textwidth}{llX}
\toprule
\textbf{Date} & \textbf{Developer(s)} & \textbf{Change}\\
\midrule
Sept 21 & Andrea Clemeno & First Draft of Problem Statement\\


\bottomrule
\end{tabularx}
\end{table}

\newpage
\section{Problem Statement}

The common cold, influenza, acquired immunodeficiency syndrome (AIDs) and the novel coronavirus (COVID-19) are infectious diseases transmitted among humans. The culprit behind these contagions are submicroscopic parasites that infect cells called viruses. Once a virus comes into contact with a host cell, the infected cell will begin to rapidly reproduce the genetic material of the virus rather than it’s own products. This phenomena causes devastating damage to the body as it may kill or modify cells. The human body’s immune system will work to counteract this attack by recognizing and implementing several methods of defense. 

Different viruses cause different responses from the human body. The human immunodeficiency virus 1 (HIV-1) is the most common type of HIV virus that attacks a type of white blood cell through direct contact with infected bodily fluids. These white blood cells (CD4 cells) are instrumental in fighting off diseases. The body’s defences are weakened by the infection causing heightened susceptibility to other diseases. The HIV-1 virus causes AIDs, a disease which affects 38 million people worldwide. 

Currently, there is no formative cure for AIDs; patients are treated with antiretroviral therapy to minimize the transmission of AIDs and the contraction of other infections. Analyzing the interaction between HIV-1 and the human body is essential in understanding and treating AIDs. This project will investigate the immunological response of the human body to HIV-1 through the use of a first-order differential equation representing the concentration of the virus over time. This simple dynamic model is based on the production and clearance rates of the virus for a specific tissue. Exploring the changing viral concentration will help determine the stage of the condition and optimal treatment. In addition, modelling the virus growth may lead to developments of a cure or better treatment options for AIDs.


\end{document}

