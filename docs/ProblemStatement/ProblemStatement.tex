\documentclass[12pt]{article}

\usepackage{tabularx}
\usepackage{booktabs}
\usepackage{hardwrap}

\usepackage[round]{natbib}

\begin{document}


\title{CAS 741: Investigation of Human 
immunodeficiency virus (HIV-1) viral load}

\author{Andrea Clemeno  400018541}

\date{\today}
\maketitle

~\newpage

%% Comments

\usepackage{color}

\newif\ifcomments\commentstrue

\ifcomments
\newcommand{\authornote}[3]{\textcolor{#1}{[#3 ---#2]}}
\newcommand{\todo}[1]{\textcolor{red}{[TODO: #1]}}
\else
\newcommand{\authornote}[3]{}
\newcommand{\todo}[1]{}
\fi

\newcommand{\wss}[1]{\authornote{blue}{SS}{#1}} 
\newcommand{\plt}[1]{\authornote{magenta}{TPLT}{#1}} %For explanation of the template
\newcommand{\an}[1]{\authornote{cyan}{Author}{#1}}


\section{Revision History}

\begin{table}[hp]
\caption{Revision History} \label{TblRevisionHistory}
\begin{tabularx}{\textwidth}{llX}
\toprule
\textbf{Date} & \textbf{Developer(s)} & \textbf{Change}\\
\midrule
Sept 21 & Andrea Clemeno & First Draft of Problem Statement\\
Sept 25 & Andrea Clemeno & Second Draft of Problem Statement\\
Sept 30 & Andrea Clemeno & Third Draft of Problem Statement\\
Dec 9 & Andrea Clemeno & Final Document of Problem Statement\\


\bottomrule
\end{tabularx}
\end{table}


\newpage
\section{Problem Statement}

The common cold, influenza, acquired immunodeficiency syndrome (AIDs) and the 
novel coronavirus (COVID-19) are infectious diseases transmitted among humans. 
The culprit behind these contagions are submicroscopic parasites that infect 
cells called viruses. Once a virus comes into contact with a host cell, the 
infected cell will begin to rapidly reproduce the genetic material of the virus 
rather than it’s own products. This phenomena causes devastating damage to the 
body as it may kill or modify cells. The human body’s immune system will work to 
counteract this attack by recognizing and implementing several methods of 
defense. 


Different viruses cause different responses from the human body. The human 
immunodeficiency virus 1 (HIV-1) is the most common type of HIV virus that 
attacks a type of white blood cell through direct contact with infected bodily 
fluids. These white blood cells (Helper T cells) are instrumental in fighting 
off 
diseases. The body’s defences are weakened by the infection causing heightened 
susceptibility to other diseases. The HIV-1 virus causes AIDs, a disease which 
affects 38 million people worldwide \citep{who}.

Currently, there is no formative cure for AIDs; patients are treated with 
antiretroviral therapy to minimize the transmission of AIDs and the contraction 
of other infections. Analyzing the interaction between HIV-1 and the human body 
is essential in understanding and treating AIDs. 

\begin{center}
$N_t$ = $N_{o} e^{-k t}$

Equation 1: The closed-form equation representing the concentration of a virus 
over time. 
\end{center}

\begin{itemize}

  \item $N_{o}$ is the initial amount that will undergo an exponential 
decrease. $N_{o}$ is equivalent to $N(t= 0)$.
  \item $N_t$ is the viral load at time t.
  \item $k$ is the clearing constant.
  \item  $t$ is the time.

\end{itemize}

This project will investigate the immunological response of the human body to 
HIV-1 through the use of the closed-form equation representing the concentration 
of a virus over time seen in Equation  1. This simple dynamic 
model is based on the exponential decline of the concentration of the virus. 
Exploring the changing viral concentration will help determine the 
stage of the condition. In addition, modelling the virus 
growth may lead to developments of a cure or better treatment options for AIDs. 
Moreover, the Drasil framework will be implemented to generate different project 
artifacts throughout the investigation. The Drasil method will be used to 
improve different aspects of software quality, including maintainability, 
verifiability, and traceability \citep{Drasilcreate}.

\newpage

\bibliographystyle{plainnat}
\bibliography{references.bib}


\end{document}

